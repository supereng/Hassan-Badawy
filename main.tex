

%----------------------------------------------------------------------------------------
%	PACKAGES AND OTHER DOCUMENT CONFIGURATIONS
%----------------------------------------------------------------------------------------

\documentclass[letterpaper]{twentysecondcv} % a4paper for A4

%----------------------------------------------------------------------------------------
%	 PERSONAL INFORMATION
%----------------------------------------------------------------------------------------

% If you don't need one or more of the below, just remove the content leaving the command, e.g. \cvnumberphone{}
% \profilepic{img/hb.jpg} % Profile picture

\cvname{Hassan Badawy} % Your name
\cvjobtitle{\begin{center} \textbf{Machine Learning Engineer}\end{center}} % Job title/career

% \cvlinkedin{hassan-badawy-1b15a374}
\cvdate{12-Sep-1986} % Date of birth
\cvnumberphone{+20 1009673933} % Phone number
\cvsite{Egypt - Cairo } % Personal website
\cvmail{hassan.badawy@outlook.com} % Email address

%----------------------------------------------------------------------------------------

\begin{document}
%----------------------------------------------------------------------------------------
%	 ABOUT ME
%----------------------------------------------------------------------------------------

 % To have no About Me section, just remove all the text and leave \aboutme{}

%----------------------------------------------------------------------------------------
%	 SKILLS
%----------------------------------------------------------------------------------------

% Skill bar section, each skill must have a value between 0 an 6 (float)
% \skills{{Arduino, Raspberry-Pi/5.5},{Scikit-Learn, Pandas/4.8},{OpenCV/5},{TensorFlow , Keras/4.5},{Matlab, LabView/4.5},{Python/5.5}}

%------------------------------------------------

\makeprofile % Print the sidebar

%%%%%%%%%%%%%%%%%%%%%%%%%%%%%%%%%%%%%%%%%h
\section{Education}

\begin{twenty} % Environment for a list with descriptions
	\twentyitem
    	{2016 - 2018 \\ (Expected)}
        {MSc. Computer Science}
    	{Benha Univ, Egypt}
        {Faculty of Engineering}
        {\textit{Thesis title: Optimization techniques for object reconstruction.}}
	\twentyitem
    	{2005 - 2010}
        {B.Sc. Communications Engineering}
        {Benha Univ, Egypt}
        {Faculty of Engineering}
        {Grade: 75\%}
	%\twentyitem{<dates>}{<title>}{<organization>}{<location>}{<description>}
\end{twenty}

%%%%%%%%%%%%%%%%%%%%%%%%%%%%%%%%%%%%%%%%%
\section{Research}
\begin{twenty}
	\twentyitem
    	{2018}
        {Discrete Grey Wolf Optimization for Shredded Document Reconstruction}       {\href{http://egyptscience.net/AISI2018/home.html}{AISI2018-Conf}}{}
        {
%         - A paper describing the ...
    }
    
	\twentyitem
    	{Under-Pub}
        {Fast Discrete Gray Wolf Optimizer for Shredded Document Reconstruction.}
        {}{}{}{}
        
	\twentyitem
    	{In-Progress}
        {PID Auto-Tuning using Optimization Techniques.}
        {}{}{}{}
        
	\twentyitem
    	{In-Progress}
        {Blind Deconvolution using Gray Wolf Optimizer..}
        {}{}{}{}
        
\end{twenty}

%%%%%%%%%%%%%%%%%%%%%%%%%%%%%%%%%%%%%%%%%

\section{Work Experience}

\begin{twenty} % Environment for a list with descriptions
    \twentyitem
	{2012-Now}   
    {\href{http://www.aou.edu.eg}{Teaching Assistant}}{Arab Open University} {}

    \twentyitem
    	{2010-2011}   
        {\href{http://www.egyptsat.com}{Satellite VSAT System Engineer}}{EgyptSat Co.} {}

\end{twenty}

%%%%%%%%%%%%%%%%%%%%%%%%%%%%%%%%%%%%%%%%%
\section{Machine Learning Projects}
\begin{twenty} % Environment for a list with descriptions
    \twentyitem
 		{2018}   {\href{https://www.kaggle.com/c/google-ai-open-images-object-detection-track}{Google AI Open Images - Object Detection Track}}{Orange Labs} {}
        
    \twentyitem
   		{2018}       {\href{https://www.kaggle.com/c/google-ai-open-images-visual-relationship-track}{Google AI Open Images - Visual Relationship Track}}
        {Orange Labs} {}
        
    \twentyitem
   		{2018}
        {\href{https://}{Shredded paper reconstruction using discrete GWO}}
        {Master Project} {} {}
        
    \twentyitem
   		{2017}
        {\href{http://www.lablineeg.com}{Water-Bath Auto-Tuning PID controller using GWO }}
        {LabLineEg Co.}
        {} {
%         - Getting best P, I, D parammeters for any system is a hard problem the old techniques like Ziegler–Nichols method has a limitation in accuracy, system complixity and time, using GWO we can overcome this limitations and introduce a fast, accurate and reliable device.
         } 
\twentyitem
   		{2016}
        {\href{https://github.com/supereng/HB-Python/blob/master/ECG3-Python.ipynb}{ECG Auto-Diagnosing System using Neural Networks and Rspberry-Pi.}}
        {Benha-Univ}
        {} {
%     - ECG device is the device that read human heart beats using 6 electrodes the features of the signal is extracted from P-R-Q-S-T and after tinning the neural network on MIT-BIH Arrhythmia ECG database we can receive new signals and predict cardiac diseases.
        }
        
\end{twenty}

%%%%%%%%%%%%%%%%%%%%%%%%%%%%%%%%%%%%%%%%%
\newpage % Start a new page
\makeprofile % Print the sidebar
%%%%%%%%%%%%%%%%%%%%%%%%%%%%%%%%%%%%%%%%%

\section{Embedded System Projects}
\begin{twenty}   
    \twentyitem{2010}{AM/FM Modulator/Demodulator simulator using LabView.}{}{}{}  
    
    \twentyitem{2010}{Design of DSP IIR/FIR digital filters using Matlab.}{}{}{}

    \twentyitem{2010}{Home Plug Network Enhancement (Graduation Project).}{}{}{}
    
    \twentyitem{2012}{Cars fuel saving system using tuned stepper motors with fuel injectors.}{}{}{}

    \twentyitem{2012}{Production line control using pneumatic and hydraulic systems (Ministry of Industry).}{}{}{}

    \twentyitem{2013}{Production line control using PLC (SIEMENS STEP 7) and Monitoring using SCADA system(Ministry of industry).}{}{}{}

    \twentyitem{2014}{Line follower robot using Arduino (Training course project).}{}{}{}

    \twentyitem{2014}{Remote Robot control by mobile phone (Training course project).}{}{}{}

    \twentyitem{2015}{Smart Home monitoring system using Arduino and IoT (Training course project).}{}{}{}

    \twentyitem{2015}{Car Doors security system using RFID and Arduino.}{}{}{}

    \twentyitem{2016}{Small CNC machine using stepper motors and Arduino.}{}{}{}
    
    \twentyitem{2016}{Datascenter room monitoring and alarm system using Raspberry-pi and IBM watson IoT node-red (Manufacturing project)}{}{}{}

\end{twenty}

%%%%%%%%%%%%%%%%%%%%%%%%%%%%%%%%%%%%%%%%%
\section{Courses \& Certificates}
\begin{twenty}   
    \twentyitem{Coursera}{\href{https://www.coursera.org/learn/machine-learning}{Introduction to Machine Learning - Andrew NG}}{}{}{}

    \twentyitem{udemy}{\href{https://www.udemy.com/machinelearning/learn/v4/t/lecture/10628128?start=0}{Machine Learning A-Z™}}{}{}{}

    \twentyitem{Udacity}
{\href{https://classroom.udacity.com/courses/ud730}{From Machine Learning to Deep Learning}}{}{}{}

    \twentyitem{Udemy}{\href{https://}{Machine learning and Deep learning}}{}{}{}
    
    
    \twentyitem{Edx}
{\href{https://courses.edx.org/courses/course-v1:Microsoft+DEV290x+2T2018/course}{Computer Vision and Image Analysis}}{}{}{}

    \twentyitem{Edx}
{\href{https://courses.edx.org/courses/course-v1:Microsoft+DEV290x+2T2018/course/}{Statistics and Probability in Data Science using Python}}{}{}{}

    \twentyitem{Lynda}{\href{https://www.lynda.com/Google-TensorFlow-tutorials/Options-loading-data/601800/647727-4.html?autoplay=true}{Building Deep Learning Applications with TensorFlow}}{}{}{}  

     \twentyitem{2018}{\href{https://www.lynda.com/Google-TensorFlow-tutorials/Building-Deep-Learning-Applications-Keras-2-0/601801-2.html}{Building Deep Learning Applications with Keras 2.0}}{}{}{}  

     \twentyitem{Lynda}{\href{https://www.lynda.com/Keras-tutorials/Neural-Networks-Convolutional-Neural-Networks-Essential-Training/689777-2.html}{Convolutional Neural Networks Essential Training}}{}{}{}  
     
     \twentyitem{2018}{ITIL V3 Foundation}{}{}{}  
     
     \twentyitem{2017}{Local TOEFL Test, With score 610.}{}{}{}  
    
\end{twenty}   
%%%%%%%%%%%%%%%%%%%%%%%%%%%%%%%%%%%%%%%%%
\newpage % Start a new page
\makeprofile % Print the sidebar
%%%%%%%%%%%%%%%%%%%%%%%%%%%%%%%%%%%%%%%%%
% \section{About Me}

%%%%%%%%%%%%%%%%%%%%%%%%%%%%%%%%%%%%%%%%%
\section{Skills}
\begin{itemize}
\item Strong understanding of machine learning techniques and algorithms, such as k-NN, Naive Bayes, SVM, Decision Forests, Logistic Regression, etc. 

\item Have an advanced knowledge of machine learning, Deep Learning, probability and statistics, IOT data streaming, NoSQL databases and Distributed computing environment.

\item Data Visualizing Info-Graphics, Excel, pandas, Seaborn, Tablue and Microsoft power-BI.

\item Working with large data and tools for data analysis, e.g., Pandas, Numpy, Scikit-learn, TensorFlow, Keras, etc.

\item Experience with Azure, Google Cloud, Amazon AWS  stack experience and working with models in the cloud.

\item Understanding of distributed systems, microservice architecture, concurrent programming

\item The ability to deal with machine learning library (Tensor Flow, Theano, Keras, PyTorch, Caffe2 etc.

\item Data Cleaning, normalising and labelling experience
 
\item Knowhow with Git, Docker, GitHub, GitLab and Kaggle platforms.

\item large-scale data analysis using  Postgres ,MongoDB, Apache Spark and a good knowledge in Hadoop.

\item Applying data mining techniques, doing statistical analysis, and building high quality prediction systems

\item Develop a toolbox and AI approaches to extract useful information from images and other data from assembly Text

\item Selecting features, building, and optimizing classifiers using machine learning techniques Data mining using state-of-the-art methods

\item Excellent practical data-analysis skills on real datasets, including familiarity with methods for working with large data and tools for data analysis, e.g., Pandas, Numpy, Scikit-learn, TensorFlow, Keras, etc.

\item Object Oriented Programming (OOP) and software design skills, preferably obtained using C++. Extensive Python experience would be a plus as would be experience with Reactive Programming

\item Familiarity with Time-Series analysis using Deep Learning, as well as experience in Reinforcement Learning would be a plus.

\item Familiarity with Linux/UNIX operating systems, including command-line.

\ The ability to Identify opportunities to use machine learning to improve data quality or the types of analysis we can offer.

\item Work side by side with other software engineers to bring your research to production and ship it as part of the product

\end{itemize}
\end{document} 
%%%%%%%%%%%%%%%%%%%%%%%%%%%%%%%%%%%%%%%%%
% Required Jobs Name
% Quantitative Researcher
% Data Scientist
% Data Engineer
% Deep Learning Research Engineer
% Full Stack Engineer
%
%%%%%%%%%%%%%%%%%%%%%%%%%%%%%%%%%%%%%%%%%
% Required Skills
 

%%%%%%%%%%%%%%%%%%%%%%%%%%%%%%%%%%%%%%%%%